\documentclass[11pt, oneside]{article}   	% use "amsart" instead of "article" for AMSLaTeX format
\usepackage[margin=1in]{geometry}                		% See geometry.pdf to learn the layout options. There are lots.
\geometry{letterpaper}                   		% ... or a4paper or a5paper or ... 
%\geometry{landscape}                		% Activate for rotated page geometry
%\usepackage[parfill]{parskip}    		% Activate to begin paragraphs with an empty line rather than an indent
\usepackage{graphicx}				% Use pdf, png, jpg, or eps§ with pdflatex; use eps in DVI mode
								% TeX will automatically convert eps --> pdf in pdflatex		
\usepackage{amssymb}
%usepackage{undertilde}
\usepackage[numbered,framed]{matlab-prettifier}

\usepackage[T1]{fontenc}
\usepackage{mathtools}  % loads »amsmath«
\usepackage{physics}
\usepackage{listings}

\lstset{
  language=C,                % choose the language of the code
  numbers=left,                   % where to put the line-numbers
  stepnumber=1,                   % the step between two line-numbers.        
  numbersep=5pt,                  % how far the line-numbers are from the code
  backgroundcolor=\color{white},  % choose the background color. You must add \usepackage{color}
  showspaces=false,               % show spaces adding particular underscores
  showstringspaces=false,         % underline spaces within strings
  showtabs=false,                 % show tabs within strings adding particular underscores
  tabsize=2,                      % sets default tabsize to 2 spaces
  captionpos=b,                   % sets the caption-position to bottom
  breaklines=true,                % sets automatic line breaking
  breakatwhitespace=true,         % sets if automatic breaks should only happen at whitespace
  title=\lstname,                 % show the filename of files included with \lstinputlisting;
}

\setlength{\parskip}{0.5em}

%SetFonts
\newcommand\Rey{\mbox{\textit{Re}}}

\title{\vspace{-6ex}\large PHYS 516: Methods of Computational Physics \\
  \normalsize ASSIGNMENT 4- Molecular Dynamics Simulation}
\author{Anup V Kanale}
\date{\vspace{-3ex}\today}							% Activate to display a given date or no date

\begin{document}
\maketitle
\vspace{-2ex} The purpose of this assignment is to get familiar with basic concepts in ordinary differential equations using a simple molecular dynamics (MD) program, \textit{md.c}, as an example.
\vspace{-2ex}
\section{Liouville Theorem}
\vspace{-2ex} Here, we show that for a particle in 1-D space, the velocity Verlet algorithm exactly preserves the phase space volume for arbitrary time discretization unit, $\Delta$.

 Let $x$ be the position of the particle, $p = mv$ the momentum where, $m$ and $v$ are the mass and velocity of the particle. In the dimensionless form, $p = v$. Let the coordinate and momentum of the particle at time $t$ be $(x, p)$, and those at time $t + \Delta$ be $(x',p')$
 \begin{align}
  x' &= x + p \Delta + \frac{1}{2} a(x) + \Delta^2 \\
  p' &= p + \frac{a(x) + a(x')}{2} \Delta
 \end{align}

To show this, it is enough to show that the Jacobian of the transformation  $(x'(x, p), p'(x, p))$ is $1$, i.e. $ J = \left| \dfrac{\partial (x', p')} {\partial (x, p)} \right| = 1$. The derivatives are calculated from Eq (1) and (2) as follows:
 \begin{align*}
 \pdv{x'}{x} &= 1 \\
 \pdv{x'}{p} &= \Delta \\
 \pdv{p'}{x} &= 0 \\
 \pdv{p'}{p} &= 1
 \end{align*}

Therefore, the Jacobian is given by
 \begin{align}
 J &= \begin{vmatrix}
   \dfrac{\partial x'}{\partial x} & \dfrac{\partial p'} {\partial x} \\[2ex]
   \dfrac{\partial x'}{\partial p} & \dfrac{\partial p'}{\partial p}
   \end{vmatrix} \\[1ex]
   &= \begin{vmatrix}
   1 & 0 \\[1ex]
   \Delta & 1
 \end{vmatrix} \\[1ex]
  &= 0
 \end{align}
\pagebreak
 
\section{Comparison of Euler and Velocity-Verlet Algorithms}
 Total energy of the system should be conserved. But due to numerical errors, energy conservation depends on the time-integration algorithm used. In this section, we illustrate this by comparing the Euler and Velocity-verlet algorithms.
\end{document}