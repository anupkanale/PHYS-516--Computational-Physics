\documentclass[11pt, oneside]{article}   	% use "amsart" instead of "article" for AMSLaTeX format
\usepackage[margin=1in]{geometry}                		% See geometry.pdf to learn the layout options. There are lots.
\geometry{letterpaper}                   		% ... or a4paper or a5paper or ... 
%\geometry{landscape}                		% Activate for rotated page geometry
%\usepackage[parfill]{parskip}    		% Activate to begin paragraphs with an empty line rather than an indent
\usepackage{graphicx}				% Use pdf, png, jpg, or eps§ with pdflatex; use eps in DVI mode
								% TeX will automatically convert eps --> pdf in pdflatex		
\usepackage{amssymb}
\usepackage{undertilde}

\usepackage[T1]{fontenc}
\usepackage{mathtools}  % loads »amsmath«
\usepackage{physics}

\setlength{\parskip}{0.5em}
%SetFonts

%SetFonts
\newcommand\Rey{\mbox{\textit{Re}}}

\title{\large PHYS 516: Methods of Computational Physics \\
  \normalsize ASSIGNMENT 1- Writing like a Computational Scientist}
\author{Anup V Kanale}
\date{\today}							% Activate to display a given date or no date

\begin{document}
\maketitle
\section{Problem}
Given a $2 \times 2$ matrix
\begin{equation}
\boldsymbol{A} = 
\begin{bmatrix}
a && b \\
b && a
\end{bmatrix}
\end{equation}
where $a$ and $b$ are real numbers, derive a closed form expression for its $n^{th}$ power $\boldsymbol{A}^n$.
\section{Solution}
Consider the Eigenvalue problem,
\begin{equation}
\boldsymbol{A} \boldsymbol{u} = \lambda \boldsymbol{u}
\end{equation}
where A is the given matrix, $\lambda$ is the eigenvalue, and $\boldsymbol{u}= \begin{bmatrix}
u \\ v
\end{bmatrix}$ is the corresponding eigenvector.
Eq(1) can be equivalently written as
\begin{align}
\boldsymbol{A} \boldsymbol{u} - \lambda \boldsymbol{u} &= 0 \\
(\boldsymbol{A}-\lambda \boldsymbol{I})\boldsymbol{u} &= 0
\end{align}
where $\boldsymbol{I}$ is the identity matrix of the same dimensions as $\boldsymbol{A}$. For nontrivial solutions ($u=v \neq 0$), the matrix on the LHS must be singular. Mathematically,
\begin{equation}
det(\boldsymbol{A}-\lambda I) = 0
\end{equation}
Substituting the matrix given in this problem for A, we get
\begin{equation}
\begin{vmatrix}
a-\lambda && b \\
b && a-\lambda \\
\end{vmatrix} = 0
\end{equation}
To obtain $\lambda$, let us evaluate this determinant.
\begin{align}
(a-\lambda)^2 - b^2 &= 0 \\
\Rightarrow \qquad \qquad (a-\lambda)^2 &= b^2 \\
\Rightarrow \quad \qquad \qquad \ a-\lambda &= \pm b \\
\Rightarrow \qquad \qquad \qquad \quad \lambda &= a \pm b
\end{align}
Therefore, there are two eigenvalues-- $\lambda_+ = a + b$ and $\lambda_- = a - b$.

Let us now evaluate the eigenvector corresponding to each of these eigenvalues. Using Eq(4)
\begin{equation}
\begin{bmatrix}
a-\lambda_+ && b\\
b && a-\lambda_+
\end{bmatrix} \begin{bmatrix}
u_+ \\ v_+
\end{bmatrix} = \begin{bmatrix}
0 \\ 0
\end{bmatrix}
\end{equation}
Plugging in the value of $\lambda_+$, we obtain the eigenvector as
\begin{equation}
\begin{bmatrix}
u_+ \\ v_+
\end{bmatrix}= \begin{bmatrix}
1 \\ 1
\end{bmatrix}
\end{equation}
Similarly, the eigenvector corresponding to $\lambda_-$ is obtained to be
\begin{equation}
\begin{bmatrix}
u_+ \\ v_+
\end{bmatrix}= \begin{bmatrix}
1 \\ -1
\end{bmatrix}
\end{equation}
Now, let us define \textbf{D} as a matrix containing the eigenvalues along it's diagonal and \textbf{U} as a matrix which contains the eigenvectors as its columns, in the same order as the eigenvalues in \textbf{D}.
\begin{align}
\boldsymbol{D} &= \begin{bmatrix}
\lambda_+ && 0 \\
0 && \lambda_-
\end{bmatrix} \\
\boldsymbol{U} &= \begin{bmatrix}
u_+ && u_- \\
v_+ && v_-
\end{bmatrix}
\end{align}
Hence, using Eq(2), the above two equations can be expressed together in matrix form as
\begin{equation}
\boldsymbol{A} \boldsymbol{U} = \boldsymbol{U} \boldsymbol{D}
\end{equation}
Right-multiplying both sides by $\boldsymbol{U}^{-1}$,
\begin{equation}
\boldsymbol{A} = \boldsymbol{U} \boldsymbol{D} \boldsymbol{U}^{-1}
\end{equation}
Therefore,
\begin{equation}
\boldsymbol{A}^n = \boldsymbol{U} \boldsymbol{D}^n \boldsymbol{U}^{-1}
\end{equation}
In this problem,
\begin{align}
\boldsymbol{U} = \begin{bmatrix}
1 && 1 \\
1 && -1
\end{bmatrix} \implies
\boldsymbol{U}^{-1} = 0.5\begin{bmatrix}
1 && 1 \\
1 && -1
\end{bmatrix}
\end{align}
Plugging in these values,
\begin{equation}
\begin{bmatrix}
a && b \\
b && a
\end{bmatrix} = 0.5 \begin{bmatrix}
1 && 1 \\
1 && -1
\end{bmatrix} \begin{bmatrix}
a+b && 0 \\
0 && a-b
\end{bmatrix} \begin{bmatrix}
1 && 1 \\
1 && -1
\end{bmatrix}
\end{equation}
Therefore,
\begin{align}
\begin{bmatrix}
a && b \\
b && a
\end{bmatrix}^n &= 0.5 \begin{bmatrix}
1 && 1 \\
1 && -1
\end{bmatrix} \begin{bmatrix}
(a+b)^n && 0 \\
0 && (a-b)^n
\end{bmatrix} \begin{bmatrix}
1 && 1 \\
1 && -1
\end{bmatrix} \\
&= 0.5\begin{bmatrix}
(a+b)^n + (a-b)^n && (a+b)^n - (a-b)^n \\
(a+b)^n - (a-b)^n && (a+b)^n + (a-b)^n
\end{bmatrix}
\end{align}
\end{document}