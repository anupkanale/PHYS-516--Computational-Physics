\documentclass[11pt, oneside]{article}   	% use "amsart" instead of "article" for AMSLaTeX format
\usepackage[margin=1in]{geometry}                		% See geometry.pdf to learn the layout options. There are lots.
\geometry{letterpaper}                   		% ... or a4paper or a5paper or ... 
%\geometry{landscape}                		% Activate for rotated page geometry
%\usepackage[parfill]{parskip}    		% Activate to begin paragraphs with an empty line rather than an indent
\usepackage{graphicx}				% Use pdf, png, jpg, or eps§ with pdflatex; use eps in DVI mode
								% TeX will automatically convert eps --> pdf in pdflatex		
\usepackage{amssymb}
\usepackage{undertilde}

\usepackage[T1]{fontenc}
\usepackage{mathtools}  % loads »amsmath«
\usepackage{physics}

\setlength{\parskip}{0.5em}
%SetFonts

%SetFonts
\newcommand\Rey{\mbox{\textit{Re}}}

\title{\large PHYS 516: Methods of Computational Physics \\
  \normalsize ASSIGNMENT 1- Writing like a Computational Scientist}
\author{Anup V Kanale}
\date{\today}							% Activate to display a given date or no date

\begin{document}
\maketitle
\section{Part 1- Theoretical Foundation of Metropolis Foundation}
Consider a set of N states, {$\Gamma_1, \Gamma_2, ..., \Gamma_N$} and let the probability to find the system in the $m$-th state, $\Gamma_m$, be $\rho_m$. Prove that the probability distribution is a fixed point of the metropolis transition matrix defined below, i.e., $\Pi \rho = \rho$.

\[
\text{(Metropolis transition matrix)} \pi_{mn}= 
\begin{dcases}
	\alpha_{mn} & \rho_m \geq \rho_n m \neq n \\
    \frac{\rho_m}{rho_n} \alpha_{mn} & \rho_m \leq \rho_n m \neq n \\
    1- \sum\limits_{m' \neq n} \pi_{m'n}
\end{dcases}
\]

Here, $pi_{mn}$ are elements of the matrix $\Pi$, $\rho_m$ are the elements of vector $\rho$, and $\alpha_{mn}$ are elements of a symmetric attempt matrix, i.e., $\alpha_{mn}= \alpha_{nm}$.
\section{Solution}
\end{document}